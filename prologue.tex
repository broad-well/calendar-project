\documentclass[letterpaper,11pt]{article}

\begin{document}
\title{The Calendar Project: Prologue}
\author{Michael Peng}
\date{June 2018}
\maketitle

With an infamous history of student dread, the legendary Calendar Project is one of those timeless concepts that would continue to haunt all of its participants indefinitely. Deriving from the feedback of Enriched Geometry alumni, it is clear that the Calendar Project is a memorable and unique experience that permanently transforms its participants. With this in mind, I feel honored to be able to engage in it as a student (i.e. victim). In a few years, I will gain the wisdom to inform those fellow Enriched Geometry students again of the fun, the horror, and the uniqueness of The Calendar Project\texttrademark.
 
The cover to this project contains multiple geometric elements that were explored throughout my year of Enriched Geometry. The word ``\texttt{CALENDAR}'' is rendered isometrically; this is covered in Chapter 10 (Spatial Reasoning), Section 2. The background is the graph of multiple equations on the two-dimensional Euclidean coordinate plane, with the origin located at the center of the page. The equations are listed as follows:


\begin{center}
\renewcommand{\arraystretch}{1.4}
\begin{tabular}{|l|l|l|l|l|}
\hline
$y=\frac{1}{x}$ & $y=-\frac{1}{x}$ & $y=\frac{4}{x}$ & $y=-\frac{4}{x}$ & $x^2+y^2=8$ \\
\hline
$y^2=x$ & $-y^2=x$ & $x^2$ & $-x^2$ & $x^2+y^2=2$ \\
\hline
\end{tabular}
\end{center}

During your adventure through the following 30 explanations, you will periodically encounter inline citations like this (1); this means that an external reference (usually to our textbook) is included to further provide rationale for the claim surrounding its inline citations. The number in the citation refers to the position of the corresponding external reference in the enumerated list that immediately follows the explanation containing the citation.
\end{document}