% !TeX spellcheck = en_US
\documentclass[letterpaper,12pt,twoside]{report}
\usepackage{fancyhdr}
\usepackage{fullpage}
\usepackage{tikz}
\usepackage{amsmath}

\begin{document}
	\pagestyle{fancy}
	\fancyhf{}
	\fancyhead[L]{Day 25}
	\fancyhead[R]{\textit{The Calendar Project}}
	\fancyfoot[L]{Citations Involved: none}
	
	% Problem
	\paragraph{Problem}
	\begin{quote}
		\textsf{The three angles of a triangle have measures $a$, $b$, and $c$, with $a<b<c$.
			Express the difference in terms of $a$, $b$,
			and $c$ when the largest interior angle is
			subtracted from the largest exterior
			angle.}
	\end{quote}
	
	% Graphics
	\begin{center}
		\begin{tikzpicture}
		% ???
		\end{tikzpicture}
	\end{center}
	
	% Reasoning
	\paragraph{Reasoning}
	\begin{quotation}
		
		Given that $a<b<c$, it is shown that $c$ is the largest interior angle of the triangle.
		
		The exterior angle that corresponds to an interior angle with measure $x$ can be expressed as $180-x$. Given that $a<b<c$, negating each term and consequently negating the comparison operators yield $-a>-b>-c$. $180-a>180-b>180-c$ when 180 is added to each term; since these values represent exterior angles, the largest exterior angle is $180-a$.
		
		The largest interior angle subtracted from the largest exterior angle is therefore $(180-a)-c$. By the Triangle Sum Theorem (1), $a+b+c=180$ and $b=180-a-c$ when $a$ and $c$ are subtracted from both sides. Since this matches the difference described in the problem ($(180-a)-c$), its solution can be expressed as $\boxed{b}$.
		
	\end{quotation}
	
	\paragraph{External References}
	
	\begin{enumerate}
		\item Textbook Ch. 4, Pg. 223: Triangle Sum Theorem
	\end{enumerate}
	
\end{document}