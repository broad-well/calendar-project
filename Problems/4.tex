% !TeX spellcheck = en_US
\documentclass[letterpaper,12pt,twoside]{report}
\usepackage{fancyhdr}
\usepackage{fullpage}
\usepackage{tikz}
\usepackage{amsmath}

\begin{document}
	\pagestyle{fancy}
	\fancyhf{}
	\fancyhead[L]{Day 4}
	\fancyhead[R]{\textit{The Calendar Project}}
	\fancyfoot[L]{Citations Involved: none}
	
	% Problem
	\paragraph{Problem}
	\begin{quote}
	\textsf{A fair six-sided die is rolled three times. If the first, second, and third outcomes are denoted $a$, $b$ and $c$, respectively, what is the probability that $a<b<c$?}
	\end{quote}
	
	% Graphics
	\begin{center}
		\begin{tikzpicture}
		\end{tikzpicture}
	\end{center}
	
	% Reasoning
	\paragraph{Reasoning}
	\begin{quotation}
	
	There are $6*6*6=6^3=216$ permutations of results from rolling a six-sided die three times (1). There are $\binom{6}{3}=\frac{6!}{(6-3)!3!}=20$ permutations of $a$, $b$ and $c$ such that $a<b<c$ because each combination of 3 different numbers from 1 to 6 (disregarding order) maps directly to \textbf{one and only one} such permutation as it has only one possible order, determined by $a<b<c$. Using $\frac{\text{favorable}}{\text{all possible}}$, the solution to this problem is represented by $\dfrac{20}{216} =\boxed{\dfrac{5}{54} \approx 9.26\%}$ (2).
	
	\end{quotation}
	
	\paragraph{External References}
	
	\begin{enumerate}
		\item Textbook Ch. 13, Pg. 870: Fundamental Counting Principle
		\item Textbook Ch. 13, Pg. 873: Combinations
		\item Textbook Ch. 13, Pg. 878: Theoretical Probability
	\end{enumerate}

\end{document}