% !TeX spellcheck = en_US
\documentclass[letterpaper,12pt,twoside]{report}
\usepackage{fancyhdr}
\usepackage{fullpage}
\usepackage{tikz}
\usepackage{amsmath}
\usepackage{textcomp}

\begin{document}
	\pagestyle{fancy}
	\fancyhf{}
	\fancyhead[L]{Day 12}
	\fancyhead[R]{\textit{The Calendar Project}}
	\fancyfoot[L]{Citations Involved: none}
	
	% Problem
	\paragraph{Problem}
	\begin{quote}
		\textsf{Victor noticed that the ratios of the
			three exterior angles of a triangle, one per
			vertex and measured in degrees, are in the
			ratio of three consecutive integers. How
			many triangles have this property when
			their exterior angles are an integral
			number of degrees?}
	\end{quote}
	
	% Graphics
	\begin{center}
		\begin{tikzpicture}
		
		\end{tikzpicture}
	\end{center}
	
	% Reasoning
	\paragraph{Reasoning}
	\begin{quotation}
		
		Let the measures of the exterior angles be $m_1$, $m_2$ and $m_3$. Since the exterior angles of a triangle always sum to 360\textdegree, $m_1+m_2+m_3=360$ (1). For these measures to remain integral when placed in a ratio of three consecutive integers, the sum of these integers must be a factor of 360. Let these consecutive integers be $x-1$, $x$, and $x+1$; as such, $x-1+x+x+1=3x$ is a factor of 360. Thus, for these integers to be integral, the factor must also be a multiple of three. All eligible factors and their corresponding ratios are presented in the table below:
		
		\begin{center}
			\begin{tabular}{r | l}
				6 & $1:2:3$ \\
				9 & $2:3:4$ \\
				12 & $3:4:5$ \\
				15 & $4:5:6$ \\
				18 & $5:6:7$ \\
				24 & $7:8:9$ \\
				30 & $9:10:11$ \\
				36 & $11:12:13$ \\
				45 & $14:15:16$ \\
				60 & $19:20:21$ \\
				72 & $23:24:25$ \\
				90 & $29:30:31$ \\
				120 & $39:40:41$ \\
				180 & $59:60:61$ \\
				360 & $119:120:121$
			\end{tabular}
		\end{center}
		
		There are $\boxed{15}$ ratios listed above.
	\end{quotation}
	
	\paragraph{External References}
	
	\begin{enumerate}
		\item Textbook Ch. 6, Pg. 384: Polygon Exterior Angle Sum Theorem
	\end{enumerate}
	
\end{document}