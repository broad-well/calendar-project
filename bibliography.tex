% !TeX spellcheck = en_US
\documentclass[letterpaper,12pt]{article}
\usepackage{fullpage}
\usepackage{fancyhdr}
\usepackage{amsmath}

\begin{document}
	\pagestyle{fancy}
	\lhead{Day \texttt{NaN}}
	\rhead{\textit{The Calendar Project}}
	
		\section*{Bibliography} \textit{Visit \texttt{mcmoo.org/calendar} for links to online sources.}
	
	\vspace{1cm}
		\begin{enumerate}
			\item \textbf{Trisecting a triangle using its medians} (Day 21) \par\texttt{http://jwilson.coe.uga.edu/emt668/EMAT6680.2000/Lehman/emat6690\\/trisecttri\%27s/whywork.html}\par This webpage shows how the medians of a triangle divide it into 3 triangles of equal area. A page from a professor at The University of Georgia is generally considered trustworthy.
			
			\item \textbf{Area under a parabola} (Day 5)
			\par\texttt{http://www.soesd.k12.or.us/SIB/files/area-under-parabola(2).pdf}\par This document shows how to solve the exact area under a parabola using Riemann sums without formal Calculus. A page from Southern Oregon Education Service District is generally considered trustworthy.
			
			\item \textbf{Limit} (Day 5)
			\par\texttt{http://mathworld.wolfram.com/Limit.html}\par This webpage formally defines the mathematical notion of a limit, which is often used in Calculus. WolframMathWorld is generally considered trustworthy.
		\end{enumerate}
\end{document}